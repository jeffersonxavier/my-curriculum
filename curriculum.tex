%%%%%%%%%%%%%%%%%%%%%%%%%%%%%%%%%%%%%%%%%
% Medium Length Professional CV
% LaTeX Template
% Version 2.0 (8/5/13)
%
% This template has been downloaded from:
% http://www.LaTeXTemplates.com
%
% Original author:
% Trey Hunner (http://www.treyhunner.com/)
%
% Important note:
% This template requires the resume.cls file to be in the same directory as the
% .tex file. The resume.cls file provides the resume style used for structuring the
% document.
%
%%%%%%%%%%%%%%%%%%%%%%%%%%%%%%%%%%%%%%%%%

%----------------------------------------------------------------------------------------
%	PACKAGES AND OTHER DOCUMENT CONFIGURATIONS
%----------------------------------------------------------------------------------------

\documentclass{resume} % Use the custom resume.cls style

\usepackage[left=0.75in,top=0.6in,right=0.75in,bottom=0.6in]{geometry} % Document margins
\usepackage[utf8]{inputenc}

\name{Jefferson Nunes de Sousa Xavier}
\address{Rua 02, Casa 12, Residencial Vitória \\ São Sebastião - DF}
\address{(61) 9947-1494 \\ jeffersonx.xavier@gmail.com}
\address{Brasileiro, Casado, 22 anos}

\begin{document}

%----------------------------------------------------------------------------------------
%	EDUCATION SECTION
%----------------------------------------------------------------------------------------

\begin{rSection}{Formação}

{\bf Universidade de Brasília} \hfill {\em Agosto de 2011 - Atualmente} \\ 
Cursando Engenharia de Software \\
10º Semestre \\

\end{rSection}

%----------------------------------------------------------------------------------------
%	WORK EXPERIENCE SECTION
%----------------------------------------------------------------------------------------

\begin{rSection}{Experiência Profissional}

\begin{rSubsection}{Senado Federal}{\em Dezembro de 2015 - Atualmente}{Estágio em Desenvolvimento de Software}{}
\item Manter e evoluir sistema interno.
\item Tecnologia Java com o framework Wicket.
\end{rSubsection}

\begin{rSubsection}{Universidade de Brasília / Ministério das Comunicações}{\em Agosto de 2013 - Novembro de 2015}{Arquiteto de Software Júnior}{}
\item Desenvolver e buscar soluções de tecnologia avançada no contexto de Arquitetura de
Software.
\item Desenvolver uma Arquitetura de Software de Referência para ser usada no Ministério
das Comunicações.
\item Desenvolver sistemas de software a partir da Arquitetura de Referência proposta (Tecnologias Grails e Java).
\end{rSubsection}

% --------------------------------------------

\begin{rSubsection}{Universidade de Brasília}{\em 2014}{Curso de Curta duração Ministrado}{}
\item Tecnologia Grails
\end{rSubsection}

\end{rSection}

%----------------------------------------------------------------------------------------
%	TECHNICAL STRENGTHS SECTION
%----------------------------------------------------------------------------------------

\begin{rSection}{Qualificações}

\textbf{Idiomas}
\begin{itemize}
	\item Inglês: Compreende e lê razoavelmente.
	\item Espanhol: Compreende e lê razoavelmente.
\end{itemize}

\textbf{Competências}
\begin{itemize}
	\item Linguagens e Frameworks: C, C++, Java, Grails, Ruby on Rails, SQL.
	\item Protocolos e APIs: XML, JSON, SOAP, REST.
	\item Banco de Dados: MySQL, PostgreSQL, Microsoft SQL Sever.
	\item Outras Competências: Linux, Git, Metodologias Ágeis, UML, Sistemas Embarcados.
\end{itemize}

\end{rSection}

\end{document}
